% Options for packages loaded elsewhere
\PassOptionsToPackage{unicode}{hyperref}
\PassOptionsToPackage{hyphens}{url}
%
\documentclass[
]{article}
\usepackage{amsmath,amssymb}
\usepackage{lmodern}
\usepackage{iftex}
\ifPDFTeX
  \usepackage[T1]{fontenc}
  \usepackage[utf8]{inputenc}
  \usepackage{textcomp} % provide euro and other symbols
\else % if luatex or xetex
  \usepackage{unicode-math}
  \defaultfontfeatures{Scale=MatchLowercase}
  \defaultfontfeatures[\rmfamily]{Ligatures=TeX,Scale=1}
\fi
% Use upquote if available, for straight quotes in verbatim environments
\IfFileExists{upquote.sty}{\usepackage{upquote}}{}
\IfFileExists{microtype.sty}{% use microtype if available
  \usepackage[]{microtype}
  \UseMicrotypeSet[protrusion]{basicmath} % disable protrusion for tt fonts
}{}
\makeatletter
\@ifundefined{KOMAClassName}{% if non-KOMA class
  \IfFileExists{parskip.sty}{%
    \usepackage{parskip}
  }{% else
    \setlength{\parindent}{0pt}
    \setlength{\parskip}{6pt plus 2pt minus 1pt}}
}{% if KOMA class
  \KOMAoptions{parskip=half}}
\makeatother
\usepackage{xcolor}
\IfFileExists{xurl.sty}{\usepackage{xurl}}{} % add URL line breaks if available
\IfFileExists{bookmark.sty}{\usepackage{bookmark}}{\usepackage{hyperref}}
\hypersetup{
  pdftitle={MOOC DATA ANALYSIS},
  pdfauthor={Akhil Raj Rajan},
  hidelinks,
  pdfcreator={LaTeX via pandoc}}
\urlstyle{same} % disable monospaced font for URLs
\usepackage[margin=1in]{geometry}
\usepackage{color}
\usepackage{fancyvrb}
\newcommand{\VerbBar}{|}
\newcommand{\VERB}{\Verb[commandchars=\\\{\}]}
\DefineVerbatimEnvironment{Highlighting}{Verbatim}{commandchars=\\\{\}}
% Add ',fontsize=\small' for more characters per line
\usepackage{framed}
\definecolor{shadecolor}{RGB}{248,248,248}
\newenvironment{Shaded}{\begin{snugshade}}{\end{snugshade}}
\newcommand{\AlertTok}[1]{\textcolor[rgb]{0.94,0.16,0.16}{#1}}
\newcommand{\AnnotationTok}[1]{\textcolor[rgb]{0.56,0.35,0.01}{\textbf{\textit{#1}}}}
\newcommand{\AttributeTok}[1]{\textcolor[rgb]{0.77,0.63,0.00}{#1}}
\newcommand{\BaseNTok}[1]{\textcolor[rgb]{0.00,0.00,0.81}{#1}}
\newcommand{\BuiltInTok}[1]{#1}
\newcommand{\CharTok}[1]{\textcolor[rgb]{0.31,0.60,0.02}{#1}}
\newcommand{\CommentTok}[1]{\textcolor[rgb]{0.56,0.35,0.01}{\textit{#1}}}
\newcommand{\CommentVarTok}[1]{\textcolor[rgb]{0.56,0.35,0.01}{\textbf{\textit{#1}}}}
\newcommand{\ConstantTok}[1]{\textcolor[rgb]{0.00,0.00,0.00}{#1}}
\newcommand{\ControlFlowTok}[1]{\textcolor[rgb]{0.13,0.29,0.53}{\textbf{#1}}}
\newcommand{\DataTypeTok}[1]{\textcolor[rgb]{0.13,0.29,0.53}{#1}}
\newcommand{\DecValTok}[1]{\textcolor[rgb]{0.00,0.00,0.81}{#1}}
\newcommand{\DocumentationTok}[1]{\textcolor[rgb]{0.56,0.35,0.01}{\textbf{\textit{#1}}}}
\newcommand{\ErrorTok}[1]{\textcolor[rgb]{0.64,0.00,0.00}{\textbf{#1}}}
\newcommand{\ExtensionTok}[1]{#1}
\newcommand{\FloatTok}[1]{\textcolor[rgb]{0.00,0.00,0.81}{#1}}
\newcommand{\FunctionTok}[1]{\textcolor[rgb]{0.00,0.00,0.00}{#1}}
\newcommand{\ImportTok}[1]{#1}
\newcommand{\InformationTok}[1]{\textcolor[rgb]{0.56,0.35,0.01}{\textbf{\textit{#1}}}}
\newcommand{\KeywordTok}[1]{\textcolor[rgb]{0.13,0.29,0.53}{\textbf{#1}}}
\newcommand{\NormalTok}[1]{#1}
\newcommand{\OperatorTok}[1]{\textcolor[rgb]{0.81,0.36,0.00}{\textbf{#1}}}
\newcommand{\OtherTok}[1]{\textcolor[rgb]{0.56,0.35,0.01}{#1}}
\newcommand{\PreprocessorTok}[1]{\textcolor[rgb]{0.56,0.35,0.01}{\textit{#1}}}
\newcommand{\RegionMarkerTok}[1]{#1}
\newcommand{\SpecialCharTok}[1]{\textcolor[rgb]{0.00,0.00,0.00}{#1}}
\newcommand{\SpecialStringTok}[1]{\textcolor[rgb]{0.31,0.60,0.02}{#1}}
\newcommand{\StringTok}[1]{\textcolor[rgb]{0.31,0.60,0.02}{#1}}
\newcommand{\VariableTok}[1]{\textcolor[rgb]{0.00,0.00,0.00}{#1}}
\newcommand{\VerbatimStringTok}[1]{\textcolor[rgb]{0.31,0.60,0.02}{#1}}
\newcommand{\WarningTok}[1]{\textcolor[rgb]{0.56,0.35,0.01}{\textbf{\textit{#1}}}}
\usepackage{longtable,booktabs,array}
\usepackage{calc} % for calculating minipage widths
% Correct order of tables after \paragraph or \subparagraph
\usepackage{etoolbox}
\makeatletter
\patchcmd\longtable{\par}{\if@noskipsec\mbox{}\fi\par}{}{}
\makeatother
% Allow footnotes in longtable head/foot
\IfFileExists{footnotehyper.sty}{\usepackage{footnotehyper}}{\usepackage{footnote}}
\makesavenoteenv{longtable}
\usepackage{graphicx}
\makeatletter
\def\maxwidth{\ifdim\Gin@nat@width>\linewidth\linewidth\else\Gin@nat@width\fi}
\def\maxheight{\ifdim\Gin@nat@height>\textheight\textheight\else\Gin@nat@height\fi}
\makeatother
% Scale images if necessary, so that they will not overflow the page
% margins by default, and it is still possible to overwrite the defaults
% using explicit options in \includegraphics[width, height, ...]{}
\setkeys{Gin}{width=\maxwidth,height=\maxheight,keepaspectratio}
% Set default figure placement to htbp
\makeatletter
\def\fps@figure{htbp}
\makeatother
\setlength{\emergencystretch}{3em} % prevent overfull lines
\providecommand{\tightlist}{%
  \setlength{\itemsep}{0pt}\setlength{\parskip}{0pt}}
\setcounter{secnumdepth}{-\maxdimen} % remove section numbering
\ifLuaTeX
  \usepackage{selnolig}  % disable illegal ligatures
\fi

\title{MOOC DATA ANALYSIS}
\author{Akhil Raj Rajan}
\date{2022-11-05}

\begin{document}
\maketitle

\hypertarget{introduction}{%
\subsection{INTRODUCTION}\label{introduction}}

The aim of this project is to analyse the data related to massive open
online course (MOOC) at Newcastle University. Basically, the data mining
process has to be implemented in this case. However, the
methodology/tool called CRISP-DM(Cross Industry Standard Process for
data mining) can help to explore the data and provide some useful
information from the raw data. CRISP-Dm is the commonly used data mining
method in such cases worldwide. It consists mainly of six steps, and the
process flow is not one-directional that is depending on the situation
some steps will be executed multiple times. This idea is clearly
explained in the below graph.

The stages of CRISP-DM and the data mining process are shown below. In
this figure, a single arrow shows a phase's dependence on the other, and
a double arrow implies a repeated process.

Business understanding, Data Understanding, Modelling, Evaluation, and
Deployment are the six phases of CRISP-DM.

\includegraphics{MOOC_Analysis_files/figure-latex/unnamed-chunk-2-1.pdf}

\hypertarget{business-understanding}{%
\subsubsection{\texorpdfstring{\textbf{Business
Understanding}}{Business Understanding}}\label{business-understanding}}

The first phase of the CRISP-DM technique is business understanding or
domain understanding. By analysing, processing, and putting various
algorithms into use, determine the sector of business that will change
into relevant information at this point. Understanding business helps
you identify problems, available resources (both hardware and people),
and goals.

\hypertarget{data-understanding}{%
\subsubsection{\texorpdfstring{\textbf{Data
Understanding}}{Data Understanding}}\label{data-understanding}}

The second stage is closely related to the first understanding. Here,
the meaning of the data in the real world is identified and conveyed to
the stakeholders properly. The idea is that the person working on this
data should understand the importance and relations of each variable in
the data before proceeding to data mining.

E.g., Primary keys, foreign keys, data type, number of observations and
meaning of each field in business

\hypertarget{data-preparation}{%
\subsubsection{\texorpdfstring{\textbf{Data
Preparation}}{Data Preparation}}\label{data-preparation}}

In this stage, Data is prepared for analysis. Raw data collected from
the source cannot be used directly, instead, it has to be prepared for
starting the data mining process. In most cases, the following steps are
carried out.

Data cleaning- e.g., removing unwanted characters like space from the
data. This step is very important because it significantly affects the
joining and comparison of the data since this space may not be visible
to the human eye at first.

Data type changing- This step is also an important part, for example, if
there is a column with string data type but on checking only numeric
values are found, then changing the data type of the field to numeric
will increase the speed of accessing the data. Additionally, some
inbuilt functions are specific to certain data types alone, so using the
original data type fails in leveraging these functions.

\hypertarget{model}{%
\subsubsection{\texorpdfstring{\textbf{Model}}{Model}}\label{model}}

This is the core stage in CRISP-DM which create a model that replicates
the real-world situation to solve the problem. Different approaches can
be used for solving a problem. Some of them are

\begin{itemize}
\item
  Simple Exploratory analysis
\item
  Principle Component Analysis
\item
  Linear Discriminant Analysis
\item
  Regression model
\item
  Decision tree
\item
  Random forest
\end{itemize}

Some times multiple models is used for solving a problem in CRISP-DM

\hypertarget{evaluation}{%
\subsubsection{\texorpdfstring{\textbf{Evaluation}}{Evaluation}}\label{evaluation}}

In this phase, different models are compared and ranked based on their
performance, algorithmic simplicity and deployment cost. The best model
is selected based on these criteria. This section includes suggestions,
recommendations and criticisms.

\hypertarget{deployment}{%
\subsubsection{\texorpdfstring{\textbf{Deployment}}{Deployment}}\label{deployment}}

The deployment stage can range in complexity from deploying a repeatable
data mining process across the company to something as simple as
producing a report. This depends on the company's goals. Many times, the
customer executes the deployment stages rather than the data analyst.
However, even if the deployment effort is handled by the analyst, it is
crucial for the customer to know upfront what steps must be taken.

The deployment phase consists of mainly our tasks. They are

\begin{itemize}
\item
  making preparations for deployment
\item
  organizing, observing, and maintenance~
\item
  presenting final findings
\item
  reviewing the outcomes
\end{itemize}

\hypertarget{business-understanding-1}{%
\subsection{BUSINESS UNDERSTANDING}\label{business-understanding-1}}

The first phase of the CRISP-DM technique is business understanding. The
important stage in this phase is understanding the business requirement
and problems. The objective of this data mining is well established and
generates a good plan for the exploration in this step.

\hypertarget{data-understanding-1}{%
\subsection{DATA UNDERSTANDING}\label{data-understanding-1}}

A massive open online course (MOOC) at Newcastle University named
``Cyber Security: Safety At Home, Online, and in Life'' has been offered
seven times to the general public using the online education platform
FutureLearn.

\begin{itemize}
\tightlist
\item
  learner\_id -- The individual ID assigned to each student enrolled in
  the course.
\item
  enrolled\_at -- This column contains the learner's enrollment date in
  the course.
\item
  unrolled\_at -- The date the learner dropped out of the programme.
\item
  role -- defines the position of the registered candidate.
\item
  fully\_participated\_at - Indicates the time the learner finished the
  course..
\item
  purchased\_statement\_at -- The purchase statement's generation date
  is indicated in this column.
\item
  gender -- displays the learner's gender who has signed up for the
  course.
\item
  country -- Indicates the country the learner selected while
  registering for~the course.
\item
  age\_range -- shows the range of registered students' ages.
\item
  highest\_educational\_level - A learner's level of education.
\item
  employment\_status -- specifies the learner's job status
\item
  employment\_area - Indicates the field in which the learned are
  employed
\item
  detected\_country -- This field contains information about the
  learner's country as determined by their network address.
\end{itemize}

There are 37296 rows in this data, It is not a big number so normal
procedures can be followed. The dimensions like employment area and
detected country can be the foreign keys to some other tables.

The primary key for this data is learners\_id as it can identify unique
observations alone.

A brief Idea about the data can be seen in the below table.

\begin{longtable}[]{@{}
  >{\centering\arraybackslash}p{(\columnwidth - 4\tabcolsep) * \real{0.3611}}
  >{\centering\arraybackslash}p{(\columnwidth - 4\tabcolsep) * \real{0.2778}}
  >{\centering\arraybackslash}p{(\columnwidth - 4\tabcolsep) * \real{0.3611}}@{}}
\toprule
\begin{minipage}[b]{\linewidth}\centering
Column Name
\end{minipage} & \begin{minipage}[b]{\linewidth}\centering
Data Type
\end{minipage} & \begin{minipage}[b]{\linewidth}\centering
Example data
\end{minipage} \\
\midrule
\endhead
learner\_id & chr & 160d6600-ea0e-4568-bfa9-5d7cd5b8e61b \\
enrolled\_dttm & POSIXct & 2016-08-10 14:28:49 \\
unenrolled\_dttm & POSIXct & 2016-08-10 14:28:49 \\
role & chr & learner \\
fully\_participated\_dttm & POSIXct & 2016-08-10 14:28:49 \\
purchased\_statement\_dttm & POSIXct & 2016-08-10 14:28:49 \\
sex & chr & male \\
country & chr & GB \\
age\_range & chr & 46-55 \\
highest\_education\_level & chr & university\_degree \\
employment\_status & chr & working\_part\_time \\
employment\_area & chr & teaching\_and\_education \\
detected\_country & chr & GB \\
run & num & 2 \\
\bottomrule
\end{longtable}

\hypertarget{data-preparation-1}{%
\subsection{DATA PREPARATION}\label{data-preparation-1}}

1.Appended all the 7 data to a single data frame after adding a column
to indicate the run for easy processing. This will helps to anlayse the
data based on run in future.

\begin{Shaded}
\begin{Highlighting}[]
\CommentTok{\#loading the all the data to a data frame}

\NormalTok{cyber\_s\_e\_1}\OtherTok{=}\NormalTok{cyber.security}\FloatTok{.1}\NormalTok{\_enrolments}
\NormalTok{cyber\_s\_e\_2}\OtherTok{=}\NormalTok{cyber.security}\FloatTok{.2}\NormalTok{\_enrolments}
\NormalTok{cyber\_s\_e\_3}\OtherTok{=}\NormalTok{cyber.security}\FloatTok{.3}\NormalTok{\_enrolments}
\NormalTok{cyber\_s\_e\_4}\OtherTok{=}\NormalTok{cyber.security}\FloatTok{.4}\NormalTok{\_enrolments}
\NormalTok{cyber\_s\_e\_5}\OtherTok{=}\NormalTok{cyber.security}\FloatTok{.5}\NormalTok{\_enrolments}
\NormalTok{cyber\_s\_e\_6}\OtherTok{=}\NormalTok{cyber.security}\FloatTok{.6}\NormalTok{\_enrolments}
\NormalTok{cyber\_s\_e\_7}\OtherTok{=}\NormalTok{cyber.security}\FloatTok{.7}\NormalTok{\_enrolments}

\CommentTok{\#adding a column "year" to all the data frames for representing the year.}

\NormalTok{cyber\_s\_e\_1}\OtherTok{=}\NormalTok{cyber\_s\_e\_1 }\SpecialCharTok{\%\textgreater{}\%} \FunctionTok{mutate}\NormalTok{(}\AttributeTok{run=}\DecValTok{1}\NormalTok{)}
\NormalTok{cyber\_s\_e\_2}\OtherTok{=}\NormalTok{cyber\_s\_e\_2 }\SpecialCharTok{\%\textgreater{}\%} \FunctionTok{mutate}\NormalTok{(}\AttributeTok{run=}\DecValTok{2}\NormalTok{)}
\NormalTok{cyber\_s\_e\_3}\OtherTok{=}\NormalTok{cyber\_s\_e\_3 }\SpecialCharTok{\%\textgreater{}\%} \FunctionTok{mutate}\NormalTok{(}\AttributeTok{run=}\DecValTok{3}\NormalTok{)}
\NormalTok{cyber\_s\_e\_4}\OtherTok{=}\NormalTok{cyber\_s\_e\_4 }\SpecialCharTok{\%\textgreater{}\%} \FunctionTok{mutate}\NormalTok{(}\AttributeTok{run=}\DecValTok{4}\NormalTok{)}
\NormalTok{cyber\_s\_e\_5}\OtherTok{=}\NormalTok{cyber\_s\_e\_5 }\SpecialCharTok{\%\textgreater{}\%} \FunctionTok{mutate}\NormalTok{(}\AttributeTok{run=}\DecValTok{5}\NormalTok{)}
\NormalTok{cyber\_s\_e\_6}\OtherTok{=}\NormalTok{cyber\_s\_e\_6 }\SpecialCharTok{\%\textgreater{}\%} \FunctionTok{mutate}\NormalTok{(}\AttributeTok{run=}\DecValTok{6}\NormalTok{)}
\NormalTok{cyber\_s\_e\_7}\OtherTok{=}\NormalTok{cyber\_s\_e\_7 }\SpecialCharTok{\%\textgreater{}\%} \FunctionTok{mutate}\NormalTok{(}\AttributeTok{run=}\DecValTok{7}\NormalTok{)}
\NormalTok{cyber\_s\_e\_all}\OtherTok{=}\FunctionTok{rbind}\NormalTok{(cyber\_s\_e\_1,cyber\_s\_e\_2,cyber\_s\_e\_3,cyber\_s\_e\_4,cyber\_s\_e\_5,cyber\_s\_e\_6,cyber\_s\_e\_7)}
\end{Highlighting}
\end{Shaded}

\hfill\break
2.Checked for duplicates and no duplicates were found

Removing duplicates is necessary because it may give false information
about the data.

\begin{Shaded}
\begin{Highlighting}[]
\FunctionTok{count}\NormalTok{(}\FunctionTok{unique}\NormalTok{(cyber\_s\_e\_all))}
\end{Highlighting}
\end{Shaded}

\begin{verbatim}
## # A tibble: 1 x 1
##       n
##   <int>
## 1 37296
\end{verbatim}

\begin{Shaded}
\begin{Highlighting}[]
\FunctionTok{count}\NormalTok{(cyber\_s\_e\_all)}
\end{Highlighting}
\end{Shaded}

\begin{verbatim}
## # A tibble: 1 x 1
##       n
##   <int>
## 1 37296
\end{verbatim}

\begin{Shaded}
\begin{Highlighting}[]
\CommentTok{\#if the values are equal then there are no duplicates}
\end{Highlighting}
\end{Shaded}

\hfill\break
3.Renamed table names to the most appropriate names.

This will help the stakeholders to get an overall idea of the data just
by viewing it. This will also help the new analysts to identify the
datatypes as well.

\begin{Shaded}
\begin{Highlighting}[]
\CommentTok{\#renaming table names }
\FunctionTok{names}\NormalTok{(cyber\_s\_e\_all)[}\FunctionTok{names}\NormalTok{(cyber\_s\_e\_all) }\SpecialCharTok{==} \StringTok{\textquotesingle{}enrolled\_at\textquotesingle{}}\NormalTok{] }\OtherTok{\textless{}{-}} \StringTok{\textquotesingle{}enrolled\_dttm\textquotesingle{}}
\FunctionTok{names}\NormalTok{(cyber\_s\_e\_all)[}\FunctionTok{names}\NormalTok{(cyber\_s\_e\_all) }\SpecialCharTok{==} \StringTok{\textquotesingle{}unenrolled\_at\textquotesingle{}}\NormalTok{] }\OtherTok{\textless{}{-}} \StringTok{\textquotesingle{}unenrolled\_dttm\textquotesingle{}}
\FunctionTok{names}\NormalTok{(cyber\_s\_e\_all)[}\FunctionTok{names}\NormalTok{(cyber\_s\_e\_all) }\SpecialCharTok{==} \StringTok{\textquotesingle{}fully\_participated\_at\textquotesingle{}}\NormalTok{] }\OtherTok{\textless{}{-}} \StringTok{\textquotesingle{}fully\_participated\_dttm\textquotesingle{}}
\FunctionTok{names}\NormalTok{(cyber\_s\_e\_all)[}\FunctionTok{names}\NormalTok{(cyber\_s\_e\_all) }\SpecialCharTok{==} \StringTok{\textquotesingle{}purchased\_statement\_at\textquotesingle{}}\NormalTok{] }\OtherTok{\textless{}{-}} \StringTok{\textquotesingle{}purchased\_statement\_dttm\textquotesingle{}}
\FunctionTok{names}\NormalTok{(cyber\_s\_e\_all)[}\FunctionTok{names}\NormalTok{(cyber\_s\_e\_all) }\SpecialCharTok{==} \StringTok{\textquotesingle{}gender\textquotesingle{}}\NormalTok{] }\OtherTok{\textless{}{-}} \StringTok{\textquotesingle{}sex\textquotesingle{}}
\end{Highlighting}
\end{Shaded}

\hfill\break
4.Data cleaning by removing unwanted spaces and removing ambiguous
values

\begin{Shaded}
\begin{Highlighting}[]
\CommentTok{\#data cleaning by removing unwanted spaces and removing ambiguous values}

\NormalTok{cyber\_s\_e\_all}\SpecialCharTok{$}\NormalTok{unenrolled\_dttm }\OtherTok{\textless{}{-}} \FunctionTok{trimws}\NormalTok{(cyber\_s\_e\_all}\SpecialCharTok{$}\NormalTok{unenrolled\_dttm, }\AttributeTok{which =} \FunctionTok{c}\NormalTok{(}\StringTok{"both"}\NormalTok{))}
\NormalTok{cyber\_s\_e\_all}\SpecialCharTok{$}\NormalTok{learner\_id }\OtherTok{\textless{}{-}} \FunctionTok{trimws}\NormalTok{(cyber\_s\_e\_all}\SpecialCharTok{$}\NormalTok{learner\_id, }\AttributeTok{which =} \FunctionTok{c}\NormalTok{(}\StringTok{"both"}\NormalTok{))}
\NormalTok{cyber\_s\_e\_all}\SpecialCharTok{$}\NormalTok{role }\OtherTok{\textless{}{-}} \FunctionTok{trimws}\NormalTok{(cyber\_s\_e\_all}\SpecialCharTok{$}\NormalTok{role, }\AttributeTok{which =} \FunctionTok{c}\NormalTok{(}\StringTok{"both"}\NormalTok{))}
\NormalTok{cyber\_s\_e\_all}\SpecialCharTok{$}\NormalTok{sex }\OtherTok{\textless{}{-}} \FunctionTok{trimws}\NormalTok{(cyber\_s\_e\_all}\SpecialCharTok{$}\NormalTok{sex, }\AttributeTok{which =} \FunctionTok{c}\NormalTok{(}\StringTok{"both"}\NormalTok{))}
\NormalTok{cyber\_s\_e\_all}\SpecialCharTok{$}\NormalTok{country }\OtherTok{\textless{}{-}} \FunctionTok{trimws}\NormalTok{(cyber\_s\_e\_all}\SpecialCharTok{$}\NormalTok{country, }\AttributeTok{which =} \FunctionTok{c}\NormalTok{(}\StringTok{"both"}\NormalTok{))}
\NormalTok{cyber\_s\_e\_all}\SpecialCharTok{$}\NormalTok{age\_range }\OtherTok{\textless{}{-}} \FunctionTok{trimws}\NormalTok{(cyber\_s\_e\_all}\SpecialCharTok{$}\NormalTok{age\_range, }\AttributeTok{which =} \FunctionTok{c}\NormalTok{(}\StringTok{"both"}\NormalTok{))}
\NormalTok{cyber\_s\_e\_all}\SpecialCharTok{$}\NormalTok{highest\_education\_level }\OtherTok{\textless{}{-}} \FunctionTok{trimws}\NormalTok{(cyber\_s\_e\_all}\SpecialCharTok{$}\NormalTok{highest\_education\_level, }\AttributeTok{which =} \FunctionTok{c}\NormalTok{(}\StringTok{"both"}\NormalTok{))}
\NormalTok{cyber\_s\_e\_all}\SpecialCharTok{$}\NormalTok{employment\_status }\OtherTok{\textless{}{-}} \FunctionTok{trimws}\NormalTok{(cyber\_s\_e\_all}\SpecialCharTok{$}\NormalTok{employment\_status, }\AttributeTok{which =} \FunctionTok{c}\NormalTok{(}\StringTok{"both"}\NormalTok{))}
\NormalTok{cyber\_s\_e\_all}\SpecialCharTok{$}\NormalTok{employment\_area }\OtherTok{\textless{}{-}} \FunctionTok{trimws}\NormalTok{(cyber\_s\_e\_all}\SpecialCharTok{$}\NormalTok{employment\_area, }\AttributeTok{which =} \FunctionTok{c}\NormalTok{(}\StringTok{"both"}\NormalTok{))}
\NormalTok{cyber\_s\_e\_all}\SpecialCharTok{$}\NormalTok{detected\_country }\OtherTok{\textless{}{-}} \FunctionTok{trimws}\NormalTok{(cyber\_s\_e\_all}\SpecialCharTok{$}\NormalTok{detected\_country, }\AttributeTok{which =} \FunctionTok{c}\NormalTok{(}\StringTok{"both"}\NormalTok{))}

\NormalTok{cyber\_s\_e\_all[}\StringTok{"unenrolled\_dttm"}\NormalTok{][cyber\_s\_e\_all[}\StringTok{"unenrolled\_dttm"}\NormalTok{] }\SpecialCharTok{==} \StringTok{""}\NormalTok{] }\OtherTok{\textless{}{-}} \StringTok{"9999{-}12{-}31 00:00:00 UTC"}
\NormalTok{cyber\_s\_e\_all[}\StringTok{"fully\_participated\_dttm"}\NormalTok{][cyber\_s\_e\_all[}\StringTok{"fully\_participated\_dttm"}\NormalTok{] }\SpecialCharTok{==} \StringTok{""}\NormalTok{] }\OtherTok{\textless{}{-}} \StringTok{"9999{-}12{-}31 00:00:00 UTC"}
\NormalTok{cyber\_s\_e\_all[}\StringTok{"purchased\_statement\_dttm"}\NormalTok{][cyber\_s\_e\_all[}\StringTok{"purchased\_statement\_dttm"}\NormalTok{] }\SpecialCharTok{==} \StringTok{""}\NormalTok{] }\OtherTok{\textless{}{-}} \StringTok{"9999{-}12{-}31 00:00:00 UTC"}
\end{Highlighting}
\end{Shaded}

\hfill\break
5.Data types were modified appropriately

This will help the data analysis in future to use some inbuilt functions
specific to each data type. For example, extracting the year value from
enrolled\_dttm will be easy using the function format().

\begin{Shaded}
\begin{Highlighting}[]
\CommentTok{\#data type changing}
\NormalTok{cyber\_s\_e\_all}\SpecialCharTok{$}\NormalTok{run}\OtherTok{\textless{}{-}}\FunctionTok{as.numeric}\NormalTok{(cyber\_s\_e\_all}\SpecialCharTok{$}\NormalTok{run)}
\NormalTok{cyber\_s\_e\_all}\SpecialCharTok{$}\NormalTok{enrolled\_dttm}\OtherTok{\textless{}{-}} \FunctionTok{as.POSIXct}\NormalTok{( cyber\_s\_e\_all}\SpecialCharTok{$}\NormalTok{enrolled\_dttm, }\AttributeTok{tz =} \StringTok{"UTC"}\NormalTok{ )}
\NormalTok{cyber\_s\_e\_all}\SpecialCharTok{$}\NormalTok{unenrolled\_dttm}\OtherTok{\textless{}{-}} \FunctionTok{as.POSIXct}\NormalTok{( cyber\_s\_e\_all}\SpecialCharTok{$}\NormalTok{unenrolled\_dttm, }\AttributeTok{tz =} \StringTok{"UTC"}\NormalTok{ )}
\NormalTok{cyber\_s\_e\_all}\SpecialCharTok{$}\NormalTok{fully\_participated\_dttm}\OtherTok{\textless{}{-}} \FunctionTok{as.POSIXct}\NormalTok{( cyber\_s\_e\_all}\SpecialCharTok{$}\NormalTok{fully\_participated\_dttm, }\AttributeTok{tz =} \StringTok{"UTC"}\NormalTok{ )}
\NormalTok{cyber\_s\_e\_all}\SpecialCharTok{$}\NormalTok{purchased\_statement\_dttm}\OtherTok{\textless{}{-}} \FunctionTok{as.POSIXct}\NormalTok{( cyber\_s\_e\_all}\SpecialCharTok{$}\NormalTok{purchased\_statement\_dttm, }\AttributeTok{tz =} \StringTok{"UTC"}\NormalTok{ )}
\end{Highlighting}
\end{Shaded}

\hypertarget{modelling}{%
\subsection{MODELLING}\label{modelling}}

The objective of this analysis is to find the answers to the business
question using CRISP-DM model. For answering the first question, a
histogram plot was generated for the different age groups that show the
frequency of enrolled candidates for each group. Then, the majority of
the age group is found using a bar plot since it is the best visual
method for this task. The procedure is given below.

age\_range= Age group, n= number of candidates under this category,
prob\_n= n/Total

1.age\_range

The records with unknown as the age\_limit were eliminated for getting
accurate model

\begin{longtable}[]{@{}lrr@{}}
\toprule
age\_range & n & prob\_n \\
\midrule
\endhead
\textless18 & 42 & 0.0104270 \\
\textgreater65 & 598 & 0.1484608 \\
18-25 & 691 & 0.1715492 \\
26-35 & 875 & 0.2172294 \\
36-45 & 608 & 0.1509434 \\
46-55 & 602 & 0.1494538 \\
56-65 & 612 & 0.1519364 \\
\bottomrule
\end{longtable}

Plotting a probability diagram for this data

\includegraphics{MOOC_Analysis_files/figure-latex/unnamed-chunk-9-1.pdf}

Comments: The above graph shows that the age group ``26-35'' is the
majority enrolled group. Additionally, the age groups 36-45,46-55,56-65
all seem to be uniformly distributed.

\hypertarget{evaluation-1}{%
\subsection{Evaluation}\label{evaluation-1}}

\begin{itemize}
\item
  The model shows that the majority of candidates are from the age group
  26-35. They can be categorized as youth people.
\item
  For the remaining age group above 35, the total number of candidates
  is approximately the same in each group.
\item
  However, only a few candidates are enrolled under the age of 18.
\item
  The probability distribution of the data can be seen as more aligned
  to uniform distribution except for the age group below 18.
\item
  The overall data spread across all the age groups greater than 18.
\end{itemize}

\hypertarget{business-understanding-cycle-2}{%
\subsection{BUSINESS UNDERSTANDING-CYCLE
2}\label{business-understanding-cycle-2}}

For further exploration, the employment status of these candidates under
the age 26-35 is analysed and its result gives the idea about how these
candidates under this age group relate to the job type. The answer to
this question gives a useful insight into data mining for business
requirements.

\hypertarget{data-understanding-2}{%
\subsection{DATA UNDERSTANDING}\label{data-understanding-2}}

Here, we are using the same data as that in the first cycle. So,
additional information about the data is not required.

\hypertarget{data-preparation-2}{%
\subsection{DATA PREPARATION}\label{data-preparation-2}}

Since there are some ``unknown'' values in the employment status field,
the rows with unknowns are filtered out based on the assumption that it
is invalid.

\hypertarget{modelling-1}{%
\subsection{MODELLING}\label{modelling-1}}

The first model showed that the majority of the candidates are from the
age group 26-35. Based on this output from cycle 2, the previous data is
filtered and provided as input for cycle -2. Here, the main focus is on
answering the next question that is ``what is the employment status of
the majority age group?''. To answer, this question, a histogram plot
showing the employment status of the age group 26-35 is used. The
process is given in the below steps.

employment\_status= employment status, n= number of candidates under
this category, prob\_n= n/Total

\begin{longtable}[]{@{}lrr@{}}
\toprule
employment\_status & n & prob\_n \\
\midrule
\endhead
not\_working & 35 & 0.0404624 \\
unemployed & 48 & 0.0554913 \\
full\_time\_student & 69 & 0.0797688 \\
working\_part\_time & 75 & 0.0867052 \\
self\_employed & 106 & 0.1225434 \\
looking\_for\_work & 136 & 0.1572254 \\
working\_full\_time & 396 & 0.4578035 \\
\bottomrule
\end{longtable}

Plotting a probability diagram regarding the candidate under the age
group 26-35 for each employment status.

\includegraphics{MOOC_Analysis_files/figure-latex/unnamed-chunk-11-1.pdf}

comments: The above-given figure show that the candidate under 26-35
(majority age group ) are full-time time employees. Also, The candidates
who are looking for a job stand second higher in number under the same
category. lastly, The number of learners who are not working attending
the course under this category is less in number.

\hypertarget{evaluation-2}{%
\subsection{Evaluation}\label{evaluation-2}}

\begin{itemize}
\item
  The majority of candidates are working as a full-time employees under
  the age group 26-35. (0.46\%)
\item
  The number of candidates who are not working under this group is
  relatively very low (4\%)
\item
  This makes sense that people who are working full time might need to
  up their skills which resulted in this variation in the data.
\end{itemize}

\hypertarget{business-understanding-cycle-3}{%
\subsection{BUSINESS UNDERSTANDING-CYCLE
3}\label{business-understanding-cycle-3}}

So far, the information about the age group and the employment status is
analysed for the enrolled candidates. However, a year-wise analysis will
give better insight into the decision making like what measures should
be taken to increase the profit. should it be urgent? etc.

To answer the above question, an year-wise analysis is used.

\hypertarget{data-understanding-3}{%
\subsection{DATA UNDERSTANDING}\label{data-understanding-3}}

Here, we are using the same data as that in the first cycle. but a new
field is added which shows the date each of the candidates is enrolled.

\hypertarget{data-preparation-3}{%
\subsection{DATA PREPARATION}\label{data-preparation-3}}

The same data is used in the cycle too. But, from the enrollment\_dttm
field year is extracted and utilized for analysis. The code for this is
shown below

\begin{Shaded}
\begin{Highlighting}[]
\CommentTok{\#Students enrolled  with year}

\NormalTok{cyber\_s\_e\_all\_yr }\OtherTok{=}\NormalTok{cyber\_s\_e\_all }\SpecialCharTok{\%\textgreater{}\%} \FunctionTok{filter}\NormalTok{(age\_range}\SpecialCharTok{==}\StringTok{"26{-}35"} \SpecialCharTok{\&}\NormalTok{ employment\_status }\SpecialCharTok{==} \StringTok{"working\_full\_time"}\NormalTok{) }\SpecialCharTok{\%\textgreater{}\%}
  \FunctionTok{mutate}\NormalTok{(}
    \AttributeTok{enrolled\_year =}\FunctionTok{format}\NormalTok{(enrolled\_dttm, }\AttributeTok{format=}\StringTok{"\%Y"}\NormalTok{),}
\NormalTok{  ) }

\CommentTok{\# The data frame contain the year of allotment of all candidate}
\end{Highlighting}
\end{Shaded}

\hypertarget{modelling-cycle-3}{%
\subsection{Modelling cycle-3}\label{modelling-cycle-3}}

Here, The input for this cycle is the output from the previous one, and
the data is compared and analysed over different years. . Now, the
enrolled\_year can be used for the year-wise analysis of the candidate
under the age category 26-35 and working full time.

The table shows the percentage of candidates under the age category
26-35 and working full-time for each year.

enrolled\_year= Year enrolled, n= number of candidates under this
category, prob\_n= n/Total

\begin{longtable}[]{@{}lrr@{}}
\toprule
enrolled\_year & n & prob\_n \\
\midrule
\endhead
2018 & 92 & 0.2323232 \\
2017 & 119 & 0.3005051 \\
2016 & 185 & 0.4671717 \\
\bottomrule
\end{longtable}

The bar plot and Pie chart of the above table are given below for better
understanding

\includegraphics{MOOC_Analysis_files/figure-latex/unnamed-chunk-14-1.pdf}

\includegraphics{MOOC_Analysis_files/figure-latex/unnamed-chunk-15-1.pdf}

Comments: The number of candidates under the age category 26-35 who are
working full time is decreasing from 2016 to 2018

\hypertarget{evaluation-3}{%
\subsection{Evaluation}\label{evaluation-3}}

The result of three CRISP-DM cycles is discussed in this section. The
questions that need to be answered are:

\begin{enumerate}
\def\labelenumi{\arabic{enumi}.}
\item
  which age group were enrolled more compared to others?
\item
  What is the employment status of the majority age group who joined in
  this online learning platform?
\item
  How does the number of enrolled candidates varies over time given that
  the candidates are under the 26-35 age group and working as a
  full-time employee?
\end{enumerate}

The first two parts were already answered, and the final result is that
the number of enrolled candidates under this mentioned condition is
decreasing from 2016 to 2018 which is not good for this organisation.
This can happen due to many reasons, but the organisation need to take
immediate decisions to reverse this graph.

\hypertarget{deployment-1}{%
\subsubsection{Deployment}\label{deployment-1}}

With the help of the first two cycles, the number /percentage of
candidates who are working as full-time employees under the age group
26-35 which is also the majority age group is found. The last cycle
deals with the year-wise analysis of the output of the second cycle
which indicates the number of candidates is decreasing in each year from
2016 to 2018.

The importance of this exploratory analysis is that it shows some
insights into the existing trends of this online learning platform
regarding its business growth. The output of this design method can be
read by a business analyst or any stakeholders and they can take
immediate actions.

The graphical reports generated by these models can be easily reported
using any data visualization tools like Power BI, Tableau etc. Since the
method follows an iteration or cycle-like strategy new information with
more clarity than the previous one is obtained which is one of the
advantages of the crisp-dm methodology. This also enables the
stakeholder to initiate the steps from scratch and improve their
explorations, data mining, analysis and exploration phases immediately
as possible since CRISP-DM stands as the base for all this process.

Here, in this example, the full potential CRISP-DM is not used since no
machine learning models were implemented. Instead if ML algorithms like
decision tree, Random forest are used we can predict some of the
responsive variables effectively, and with each cycle, the prediction
accuracy will be improved.

\end{document}
